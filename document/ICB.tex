\documentclass[a4paper]{article}
\usepackage{CJK}
\usepackage{geometry}
\geometry{left=3.5cm,right=3.5cm,top=3.5cm,bottom=3.5cm}
\usepackage{fancyhdr}
\usepackage{cite}
\usepackage[colorlinks,linkcolor=red]{hyperref}
\usepackage{graphicx}
\usepackage{multirow} 
\usepackage{array}

\title{Experiment of Tattoo Detection Based on CNN and Remarks on the NIST Database}
\author{XU QINGYONG\\E-mail:N1905179E@e.ntu.edu.sg, xyongle@163.com}

\begin{document}

		
		
		\pagestyle{fancy} 
		\lhead{Note for ICB} \rhead{Author:Xu Qingyong}
		\setcounter{page}{1}
		\thispagestyle{empty}
		\maketitle
		
%		\tableofcontents
%		\newpage
%		
%		\setcounter{page}{1}
		
\section{Introduction}
	This document introduces the experiment.
	 
	\begin{itemize}
		\item Fold code: this is the experiment code. You can refer to document./code/cuda-convnet2/README.txt if you want to use it.
		\item Fold Data: this is all data of experiment. It includes Flickr data-set.
	\end{itemize}
\section{Data-set}
	Flickr dataset is downloaded from www.flickr.com.
	 This includes Flickr2349 with 2349 images, Flickr3500 with 3500 images, Flickr5000  with 5000 images and Flickr10000 with 10000 images.\\
	 Every data-set includes five groups and the ratio between tattoo and non-tattoo is same with NIST.
	

	
\section{CODE}
	
	\begin{itemize}
		\item 1, Install cuda-convnet. \\
		The version is cuda-convnet2. The detail installed command is showed in the web:
		https://code.google.com/archive/p/cuda-convnet2/.
		\item 2, convent the jpg data-set to the format of cuda-convnet. \\
		The code is showed in ./make-data/ICB\_makedata.py and the code to generate the meta 
		file is showed with code ./ICB\_makemeta.py
		\item 3. Net define. \\  The net file is located ./layers/
		\item 4. Train. \\ The train commond is showed in the fold ./xyongle\_sh
		
	\end{itemize}

	
	



\newpage
\begin{thebibliography}{999}
%	\bibitem{2015Ngan2}Ngan, M., Quinn, G. W., Grother, P. (2015). Tattoo Recognition Technology-Challenge (Tatt-C) Outcomes and Recommendations.
	
\end{thebibliography}
\end{document}
